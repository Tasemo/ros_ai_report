%----------------------------------------------------------------------------------------
%	VALUES FOR THE REPORT
%----------------------------------------------------------------------------------------

\newcommand{\name}{Tim Oelkers} % Author name
\newcommand{\matrikelnr}{s0568352} % Author name
\newcommand{\thesistitle}{AI in der Robotik} % Title of the thesis
\newcommand{\submissiondate}{XX. Month, 2021} % Submission date "Month, year"
\newcommand{\supervisor}{Patrick Baumann, M.Sc.} % Supervisor name
%\newcommand{\cosupervisor}{} % Co-Supervisor name, comment this line if there is none
\newcommand{\bibliographytitle}{Bibliography}

%----------------------------------------------------------------------------------------
%	YOUR PACKAGES (be careful of package interaction)
%----------------------------------------------------------------------------------------

\usepackage{amsthm,amsmath,amssymb,amsfonts,bbm}% Math symbols
\usepackage[onehalfspacing]{setspace}% 1.5pt line-space
\usepackage[stretch=10]{microtype}%smoother letters
\usepackage{tikz}%vector graphics tikz
\usepackage{wrapfig}% text around figures
\usepackage{subfigure}%graphics including subgraphics
\usepackage{float}%support of H positioning (no automatic placing of elements)
\usepackage{listings}%source code highlighting

%----------------------------------------------------------------------------------------
%	YOUR DEFINITIONS AND COMMANDS
%----------------------------------------------------------------------------------------

% New Commands
\newcommand{\bea}{\begin{eqnarray}} % Shortcut for equation arrays
\newcommand{\eea}{\end{eqnarray}}
\newcommand{\e}[1]{\times 10^{#1}}  % Powers of 10 notation

% Defining a theorem box for Criteria
\newtheorem{critere}{Criterion}
\newcommand{\crit}[2]{
\begin{center}  
\fbox{ \begin{minipage}[c]{0.9 \textwidth}
\begin{critere}
\textbf{\textup{ #1}} --- #2
\end{critere}
\end{minipage}  } \end{center}
}

\renewcommand*{\labelalphaothers}{} % remove + from references when citating
\renewcommand{\lstlistlistingname}{Source Code Content} % add chapter 'Source Code Conetent'

% additinal colors for coding adaptions
\definecolor{Ao}{rgb}{0.0, 0.39, 0.0}
\definecolor{antiquefuchsia}{rgb}{0.57, 0.36, 0.51}
\definecolor{bostonuniversityred}{rgb}{0.8, 0.0, 0.0}
% coding adaptions for better visibility (can be overwritten)
\renewcommand{\lstlistingname}{Code snippet}
\lstset{
	language=python,
	numbers=left,
	columns=fullflexible,
	aboveskip=5pt,
	belowskip=10pt,
	basicstyle=\small\ttfamily,
	backgroundcolor=\color{black!5},
	commentstyle=\color{black},
	morecomment=[s]{.s}{hape}, % workaround to exclude ".shape"
	keywordstyle=\color{antiquefuchsia},
	otherkeywords={self, else},
	emphstyle=\color{bostonuniversityred},
	emph={shape,units,name,filters,kernel_size,activation,padding,kernel_initializer,kernel_regularizer,bias_initializer,inputs,outputs,mode,distribution,minval,maxval, callback_args, low, high, limit, window_length, size, mu, theta, sigma},
	stringstyle=\color{Ao},
	showspaces=false,
	showstringspaces=false,
	showtabs=false,
	xleftmargin=16pt,
	xrightmargin=0pt,
	framesep=5pt,
	framerule=1pt,
	frame=leftline,
	rulecolor=\color{black},
	tabsize=2,
	breaklines=true,
	breakatwhitespace=true
}
