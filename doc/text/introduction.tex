\chapter{Introduction}

\section{Motivation}
Speech recognition systems gained immense popularity over the last years with almost every big tech company (Amazon, Google, Apple, Microsoft, etc.)
developing their own systems. And since spoken language is very complex and intend can be given in many different ways, these systems are hugely complex
and rely on the power of the cloud for evaluation. While these systems require huge amount of ressources and expertise to develop, command recognition 
is a simpler subset of speech recognition that should achievable more realisticly. It is also important, that all current speech recognition systems listen
for a specific command ("Hey Google", "Alexa", etc.) as a trigger to start analyzeing, so its usefullness is still present in these complex sytems.
But still, human voice is incredibly diverse and even the voice of one speaker or pronounciations of specific commands can change very rapidly, so the
usefulness of these simpler systems has to evaluated in the first place. And if successful, this projects can help to cut the dependency on data collecting companies
by running the command recognition locally.

\section{Project Objective}
This projects aims to develop a human command recognition system using machine learning. The scope should be limited to
simple and short commands and its usefulness should be evaluated in a realistic, imperfect environment. This includes testing 
the model with a real microphone in a noisy environment with multiple speakers. It should be also evaluated for multiple people
with different voice tones. The system should be able to run on a middle class computer. The main evaluation aspect is the accuracy of the trained
model, the training itself (e.g. training time) should not be evaluated.

\section{Proceeding and Structure of the Work}
The first step will be focused on research on the field of command recognition. A specific type of machine learning has to be selected and
the design has to be considered before the implementation. Ideally, a existing model has already been developed and this projects builds on top of it.
The implementation will focus on two different parts. The first one on the training of the model and the second one on the building
of an evaluation environment and the evaluation itself. Here, a libary for getting audio data from the microphone has to be found. The data found
during the evaluation has to be presented in an intuitive and expressive way.