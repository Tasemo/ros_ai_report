\chapter{Fundamentals}

\section{Robot Operating System}
ROS (Robot Operating System) is an open-source meta operating system. It runs on top of Unix based systems and 
provides abstraction over process management, low-level communication and hardware. It also provides tools and 
libraries to develop applications in the field of robotics and distributed systems. ROS is build on the concept 
of creating a (possibly distributed) peer-to-peer network of nodes. A node is a single, modular process that 
should perform a dedicated task, which is implemented using C++ (roscpp) or Python (rospy). All nodes register 
themselfs at the ROS master, which serves as a nameserver and processer manager to the rest of the network. 
Nodes communicate which each other using ROS messages, which are data structures that define 
the type of data that is passed. The communication is usually based on ROS topics, 
which implement an asynchronous publish-subscribe pattern so that nodes are truly decoupled. 
If direct, synchronous communication with an immediate response is needed, ROS services can be used.

\section{Neural Networks}
Neural networks are a emulation of a human brain with inter-connected nodes (neurons) with weighted edges (synapses). 
The learning part comes from the process of altering each weight after each evaluation to better map inputs to their 
expected outputs. This is called supervised learning and inputs with known outputs have to be available for that.
During the forward pass, the data passes through the network while the weights are applied. To achieve non-linear mappings, 
an activation function is applied to inputs between each layer. After that, the error to the expected value is 
calculated with a predefined loss function and the weights are updated accoring to a predefined optimizer in a process 
called back-propagation. The process is repeated for a certain number of epochs so that the output gets closer and closer 
to the expected value. After the training, the model can be used on unknown input.

\subsection{Fully Connected Neural Network}
A fully connected network is a type of network where each node of one layer is connected to each node on the following layer.
It is often divided into a input layer, multiple hidden layer and one output layer, which has to have the same number
of nodes as the amount of classes in the classification.

\subsection{Convolutional Neural Networks}
Convolulation neural networks are often used in classification problems and are a specialisation of full connected neural networks. They are escpecially used 
in image and audio classification. It tries to find features or patterns in the provided data by removing noise and extracting data relevant for the actual target. 
They are constructed with a convolutional layer and a pooling layer. The convolutional parts moves a kernel filter over the input matrix and calculates the inner product of
surrounding datapoints. This creates an overlap and helps neighboring neurons react to similiar data (e.g. similiar frequencies in audio data). In the pooling layer, 
the amount of data gets reduced, for example by using max-pooling, a process where only the most active neuron in a certain area is kept and the rest being discarded.
This prevents the model from overfitting (valueing data not actualy relevant for the classification task) and improves training times. The data, now reduced to
its features, is then passed through a fully connected layer for the actual classification.
